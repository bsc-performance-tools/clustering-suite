% \long\def\greybox#1{%
%     \newbox\contentbox%
%     \newbox\bkgdbox%
%     \setbox\contentbox\hbox to \hsize{%
%         \vtop{
%             \kern\columnsep
%             \hbox to \hsize{%
%                 \kern\columnsep%
%                 \advance\hsize by -2\columnsep%
%                 \setlength{\textwidth}{\hsize}%
%                 \vbox{
%                     \parskip=\baselineskip
%                     \parindent=0bp
%                     #1
%                 }%
%                 \kern\columnsep%
%             }%
%             \kern\columnsep%
%         }%
%     }%
%     \setbox\bkgdbox\vbox{
%         \pdfliteral{0.85 0.85 0.85 rg}
%         \hrule width  \wd\contentbox %
%                height \ht\contentbox %
%                depth  \dp\contentbox
%         \pdfliteral{0 0 0 rg}
%     }%
%     \wd\bkgdbox=0bp%
%     \vbox{
%       \hbox to \hsize{
%         \box
%         \bkgdbox
%         \box
%         \contentbox
%       }
%     }%
%     \vskip\baselineskip%
% }

\definecolor{shade}{HTML}{D4D7FE}
% \definecolor{figureshade}{HTML}{D4D7FE}
% \definecolor{figureshade}{gray}{0.80}
\definecolor{figureshade}{gray}{.7}

\long\def\innercite#1
{
  \vspace{10pt}
  \noindent
  \begin{tikzpicture}
    \node [fill=shade,rounded corners=5pt]
    {
      \begin{tabular}{p{.03\columnwidth}p{.89\columnwidth}}
        ~\cite{#1} & ~\bibentry{#1}
      \end{tabular}
    };
  \end{tikzpicture}
  \vspace{-10pt}
}
% \long\def\innercite#1
% {
% \begin{tabular}{p{.03\columnwidth}p{.89\columnwidth}}
% ~\cite{#1} & ~\bibentry{#1}
% \end{tabular}
% }

% \newcommand{\floatbox}[1]
% {
% {
%   \vspace{10pt}
%   \noindent
%   \begin{tikzpicture}
%     \node [fill=figureshade,rounded corners=5pt]
%     {
%       #1
%     };
%   \end{tikzpicture}
%   \vspace{-10pt}
% }
% }

% ROTATED COLUMN HEADER
\newcommand\rotcolhead[1]{\begin{sideways}\textbf{\makecell[l]{#1}}\end{sideways}} 

% REGULAR COLUMN HEADER WITH BREAKLINES
\newcommand\colhead[1]{\textbf{\makecell[r]{#1}}} 


\newcommand\myVSpace[1][10pt]{\rule[\normalbaselineskip]{0pt}{#1}}

\def\verbatimboxed#1
{
  \begingroup
  \def\verbatim@processline
  {
    {
      \setbox0=\hbox{\the\verbatim@line}
      \hsize=\wd0
      \the\verbatim@line\par
    }
  }
  
  \setbox0=\vbox{
    \parskip=0pt\topsep=0pt\partopsep=0pt\verbatiminput{#1}
  }%
  
  \begin{center}\fbox{\box0}\end{center}%
  \endgroup
}

\newcolumntype{d}[1]{D{.}{\cdot}{#1}}

\newcommand{\ra}[1]{\renewcommand{\arraystretch}{#1}}

\newcolumntype{*}{>{\global\let\currentrowstyle\relax}}
\newcolumntype{^}{>{\currentrowstyle}}
\newcommand{\rowstyle}[1]{\gdef\currentrowstyle{#1}%
  #1\ignorespaces
}

\colorlet{tableheadcolor}{gray!25} % Table header colour = 25% gray
\newcommand{\headcol}{\rowcolor{tableheadcolor}} 

\newcommand{\rowcol}{\rowcolor{tablerowcolor}} %
    % Command \topline consists of a (slightly modified) \toprule followed by a \heavyrule rule of colour tableheadcolor (hence, 2 separate rules)
\newcommand{\topline}{\arrayrulecolor{black}\specialrule{0.1em}{\abovetopsep}{0pt}%
            \arrayrulecolor{tableheadcolor}\specialrule{\belowrulesep}{0pt}{0pt}%
            \arrayrulecolor{black}}
    % Command \midline consists of 3 rules (top colour tableheadcolor, middle colour black, bottom colour white)
\newcommand{\midline}{\arrayrulecolor{tableheadcolor}\specialrule{\aboverulesep}{0pt}{0pt}%
            \arrayrulecolor{black}\specialrule{\lightrulewidth}{0pt}{0pt}%
            \arrayrulecolor{white}\specialrule{\belowrulesep}{0pt}{0pt}%
            \arrayrulecolor{black}}
    % Command \rowmidlinecw consists of 3 rules (top colour tablerowcolor, middle colour black, bottom colour white)
\newcommand{\rowmidlinecw}{\arrayrulecolor{tablerowcolor}\specialrule{\aboverulesep}{0pt}{0pt}%
            \arrayrulecolor{black}\specialrule{\lightrulewidth}{0pt}{0pt}%
            \arrayrulecolor{white}\specialrule{\belowrulesep}{0pt}{0pt}%
            \arrayrulecolor{black}}
    % Command \rowmidlinewc consists of 3 rules (top colour white, middle colour black, bottom colour tablerowcolor)
\newcommand{\rowmidlinewc}{\arrayrulecolor{white}\specialrule{\aboverulesep}{0pt}{0pt}%
            \arrayrulecolor{black}\specialrule{\lightrulewidth}{0pt}{0pt}%
            \arrayrulecolor{tablerowcolor}\specialrule{\belowrulesep}{0pt}{0pt}%
            \arrayrulecolor{black}}
    % Command \rowmidlinew consists of 1 white rule
\newcommand{\rowmidlinew}{\arrayrulecolor{white}\specialrule{\aboverulesep}{0pt}{0pt}%
            \arrayrulecolor{black}}
    % Command \rowmidlinec consists of 1 tablerowcolor rule
\newcommand{\rowmidlinec}{\arrayrulecolor{tablerowcolor}\specialrule{\aboverulesep}{0pt}{0pt}%
            \arrayrulecolor{black}}
    % Command \bottomline consists of 2 rules (top colour
\newcommand{\bottomline}{\arrayrulecolor{white}\specialrule{\aboverulesep}{0pt}{0pt}%
            \arrayrulecolor{black}\specialrule{\heavyrulewidth}{0pt}{\belowbottomsep}}%
\newcommand{\bottomlinec}{\arrayrulecolor{tablerowcolor}\specialrule{\aboverulesep}{0pt}{0pt}%
            \arrayrulecolor{black}\specialrule{\heavyrulewidth}{0pt}{\belowbottomsep}}%

\newcommand{\mypow}[2]{\textnormal{#1}\ensuremath{\times\textrm{10}^\textrm{{#2}}}}
\newcommand{\sps}[1]{\ensuremath{^{\textrm{#1}}}}
\newcommand{\sbs}[1]{\ensuremath{_{\textrm{#1}}}}
